%% Plato FSM Translation Tech Memo
%% Abstract
\begin{abstract}
Asynchronous circuits are becoming increasingly important in system design for Internet-of-Things, where
they orchestrate the interface between big synchronous computation components and the analogue environment,
which is inherently asynchronous and has high uncertainty with respect to power supply, temperature
and long-term ageing effects. However, wide adoption of asynchronous circuits by industrial users is hindered
by a steep learning curve for asynchronous control models, such as Signal Transition Graphs, that are developed
by the academic community for specification, verification and synthesis of asynchronous circuits.
Finite State Machines are a common model type used in many applications, including desiging computer-based systems  and 
in many cases, a Finite State Machine may provide insight into the intricacies of signal interactions in an asynchronous circuit.
They are also more recognisable for industrial designers than Signal Transition Graphs are, for example. 

Previously, we have introduced a novel high-level description language for asynchronous circuits, which
is based on behavioural concepts - high-level descriptions of asynchronous circuit requirements, that can be
shared, reused and extended by users. We have also introduced an algorithm to automatically translate these to 
Signal Transition Graphs for further processing by conventional asynchronous and synchronous EDA tools. 
In this document, we aim to introduce an algorithm for translating concepts to Finite State Machines, which will 
allow a user to view the complexities that may be abstract in a Signal Transition Graph, without having to change the
concept specification in anyway. Both of these algorithms are implemented in the open-source tool, \noun{Plato}. This
tool is also integrated into the open-source tool-suite \noun{Workcraft}, providing several other back-end tools for 
a streamlined design process for asynchronous circuits. 

\end{abstract}
