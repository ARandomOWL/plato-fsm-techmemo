%% Plato FSM Translation Tech Memo
%% Conclusion
\section {Conclusions \label {sec:conclusions}}

This document contains an algorithm for translating concept specifications,
describing the behaviours of the signals in an asynchronous system, to Finite
State Machines. We discuss Finite State Machines, and a form of FSMs that is 
used more specifically for the purpose of detailing the interactions between 
signals in a circuit, Finite State Transducers. We compare the differences in how
FSMs and STGs model concurrency, and the benefits that STGs provide in this 
case. We work through the algorithm with the example of a C-Element with an 
environment. Finally, we briefly discussed a design flow using this algorithm. 

This concept to FSM translation algorithm, as well as the algorithm to translate 
concepts to STGs is implemented in an open-source tool, \noun{Plato}. This tool 
also features the domain-specific language of concepts, including several built-in 
concepts providing some signal-, gate- and protocol-level concepts. This tool 
allows a user to define their own concepts, and re-use them in multiple concept 
specifications by importing the file they are in into this specification. 

\noun{Plato} provides several features to help improve the design-flow of 
asynchronous circuits, which up until know has often involved a blank page to 
start a design, unable to reuse any structures from previous designs which may 
be useful. This reuse can cause a reduced design time for asynchronous circuits, 
which in turn can make them more favourable. 

Providing the ability to translate concepts to FSMs allows a user to view the 
intricacies of a system, possible orderings of signals and how states in the 
system may be affected by signal changes, which may be desirable to an 
industrial designer, who may understand a modelling formalism such as an FSM
more than an STG. This is reccomended only for small scenarios of snippets of 
such as a design, as a larger system can make for many possible states, which 
can take a long time to translate due to state explosion. 

However, concept specifications themselves cannot be verified and synthesized 
and creating tools for this purpose is a time consuming endeavour. 
FSMs are also not the preferred model to try verification and synthesis on. 
Therefore, it is preferable to use STGs, which feature many tools for 
such operations. Thus, concepts can be freely translated to either FSMs or STGs
without editing the concepts. 

\noun{Plato} is available from~\cite{2017_plato_github}, and is also integrated 
into, and provided in the download of \noun{Workcraft}, available 
from~\cite{Workcraft_website} an open-source tool 
providing a GUI to author and translate concepts, visualise the resulting models, 
and use them with the existing tools for verification and synthesis, further 
increasing the ease and speed of design.
